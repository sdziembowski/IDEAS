\documentclass{article}

\usepackage{url}

\usepackage[backend=bibtex]{biblatex}
\addbibresource{plan,erc}


\usepackage[bitstream-charter]{mathdesign}
\usepackage[T1]{fontenc}

%\usepackage{fullpage}

\usepackage{titling}
\newcommand{\subtitle}[1]{%
	\posttitle{%
		\par\end{center}
	\begin{center}\large#1\end{center}
	\vskip0.5em}%
}

\title{\textbf{Blockchain and Smart Contract Group\\ IDEAS NCBR}}

\subtitle{\textbf{Research Plan}}

\date{\today}

\author{Stefan Dziembowski}

\begin{document}
	
\maketitle

\section{Introduction}


\emph{Distributed cryptography} is a technology developed since the 1980s \cite{DBLP:conf/focs/Yao86,DBLP:conf/stoc/GoldreichMW87,DBLP:conf/crypto/ChaumCD87,DBLP:conf/stoc/Ben-OrGW88} building upon the ideas from the area of distributed algorithms such a the \emph{Byzantine consensus} \cite{DBLP:journals/toplas/LamportSP82}.  Its main advantage is that it reduces the trust assumptions in network environments: in the constructed protocols typically only a certain fraction of participants needs to be honest. This is in contrast with the \emph{centralized} (i.e.: ``non-distributed'') solutions where the users usually need to entirely trust a single participant (often called the \emph{trusted server}). Unfortunately, for a long time, distributed cryptography has not been applied  widely in practice, mostly due to the fact that it introduces a large computational overhead compared to the centralized solutions. In other words: typical users where not ready to pay (in terms of computing time, equipment costs, etc.) for the security benefits of this technology.

This situation has  recently dramatically changed. One reason for this was a general erosion of trust in single-server solutions, partly due to Edward Snowden's revelations \cite{enwiki:1029775076}, and partly because of cases like the \emph{Facebook-Cambridge Analytica data scandal} \cite{enwiki:1029165846} which led to understanding that the centralized solutions give too much power to the servers' operators. Another major reason of this change was the introduction of \emph{Blockchain technology} around one decade ago in \cite{nakamoto2008bitcoin}. 

What is typically understood by this name are cryptographic protocols for achieving large-scale consensus. They can work in the so-called \emph{permissionless} settings, where the the set of participants is not determined a priori, or in \emph{permissioned} settings, where the consensus is maintained by predefined groups of servers. 


The initial applications of these technologies were within the financial sector, mostly in the creation of new virtual \emph{cryptocurrencies} (such as \emph{Bitcoin} \cite{nakamoto2008bitcoin}). However, currently this technology is believed to have many more applications, in particular in schemes for managing digital identity, mortgages, land title recording, supply chain monitoring, insurance, 
clinical trials, copyright management, running decentralized organizations, energy trading, and in the Internet-of-Things \cite{econ3,forbes,econ2,Underwood:2016:BBB:3013530.2994581,EY,ibm,Microsoft5,DHL,chamber,capgemini,pwc,EY,DHL,iot,economist-dao,pwc2}. Many of such proposals are actually based on the \emph{permissioned} blockchains, which makes them much cheaper and environmentally-friendly than those based on the Bitcoin-type permissionless blockchains (see also Sec.~\ref{sec:PoSpace}). Some of these applications involve the so-called \emph{smart contracts} \cite{Szabo}, which are self-executable agreements resembling legal contracts, written in a programming language. Probably the best-known example of a blockchain platform that permits deployment of such contracts is \emph{Ethereum} \cite{Ethereum}. 

Of course, several of the aforementioned blockchain applications may be slightly too far fetched and may lose competition with the centralized solutions. Yet, we are confident that some of them will be successful, and that this technology has a substantial potential in the new digital economy. We also think that the idea of decentralization of the IT technology should be particularly attractive for the EU countries and Poland in particular, since currently all major centralized services on the Internet (e.g.: Google, Amazon, Dropbox, Alibaba, Huawei) are not operated by the European companies, giving corporations from non-European countries a competitive advantage on the global market. Hence, pushing for more decentralization is also essential from the point of view of our economic interests.

\section{Research plan}

The main goal of the \emph{Blockchain and Smart Contract Group} will be to contribute to the practical development of new protocols for blockchains, smart contracts, and a construction of new algorithms for user-blockchain interaction. 
We will to a large extent build upon the results of the basic research resulting from the past and on-going grants of the PI at the University of Warsaw: the ERC AdG grant \emph{Smart-Contract Protocols: Theory for Applications} (2021-25) and the NCN Opus grants \emph{Foundations of Cryptocurrencies} (2015-19) and \emph{Blockchain wallets -- cryptographic theory and applications} (2020-24). This is will done by moving the ideas developed within these projects closer to real-life applications (hence: there will be no thematic overlap between the work done in IDEAS and within these grants). For a moment we do not plan to transfer the on-going grants from the University of Warsaw to IDEAS, however this plan may change. One attractive option that we will consider is to apply for an ERC \emph{Proof-of-Concept} grant.

In more detail, our work will be divided in the following main topics. Of course, there will be a strong interaction between all of them, and typically the group members will be working on more than one of these topics. We also plan to dynamically adapt to the rapidly changing landscape of this area and come up with new related research tasks.




\subsection{Practice-oriented formal modeling and verification of smart contracts protocols}\label{sec:verif}

One of the main problems in this area is that the decentralized solutions are typically more complex and error-prone than the centralized ones. In particular, errors in smart contracts can lead to considerable financial losses (see, e.g., \cite{DAO}). Furthermore, other blockchain algorithms in the past had serious errors that could be used to steal large amounts of money (see, e.g., \cite{HACKETT}).  We will address these problems using tools from formal methods, in particular proof assistants and provers such as Coq \cite{Chlipala2013}, Easycrypt \cite{Barthe2013}, Why3 \cite{Santos2015}, and others. Theoretical aspects of this work are one of the topics of the on-going PI's ERC grant. In IDEAS, we will work on more technological aspects, especially on working on making this approach usable in real life by blockchain developers.

\subsection{Decentralized Finance}

One of the main problems of cryptocurrencies is their unstable exchange rate with some standard currency (the so-called \emph{fiat money}). The term \emph{Decentralized Finance} \cite{enwiki:1030159409} (often abbreviated as: \emph{DeFi}) refers to blockchain solutions that address this problem. The most notable examples are the \emph{stable coins} \cite{Clark2020}, where the internal currencies of a given blockchain are exchangeable with the fiat money at a fixed rate. There are several interesting proposals on how to construct such coins, most of them coming from blockchain startups and lacking full formal security analysis. We will work on improving these protocols and understanding their security properties (possibly using the tools developed in \ref{sec:verif}).
	
DeFi protocols can also be used as a replacement for financial institutions such as the stock market. One of the problems with this approach are the so-called \emph{front-running} attacks \cite{Eskandari2019}, where the miners (or other powerful users) constantly monitor the messages sent to the blockchain and choose their transactions depending on the transactions of the other users. Unfortunately, most of the blockchain solutions allow such powerful participants to publish their place their transactions on the blockchain \emph{before} the original transactions appear there. This can lead to considerable financial gains of such of the honest participants. We will work on addressing this problem in real life using tools from cryptography such as cryptographic commitment schemes (see, e.g., \cite{Goldreich2001}) and timed-release cryptography \cite{10.5555/888615}.

\subsection{Proofs-of-Space in practice}\label{sec:PoSpace}

The most popular blockchain platforms (including the Bitcoin one) use consensus based on the so-called \emph{Proofs-of-Work} \cite{Dwork1992}, where the participants are incentivized to constantly solve a large amount of computational ``puzzles'' (this process is also called \emph{mining}). This leads to massive electricity consumption. It is currently estimated that the Bitcoin blockchain alone wastes more electricity than a mid-sized country \cite{Criddle}. Several alternatives to Bitcoin mining have been proposed in the past. The most notable ones are probably the \emph{Proof-of-Stake} approach, see e.g.~\cite{Kiayias2017} (another option is to switch to permissioned blockchains).

The PI is one of the authors of another approach to this problem, namely the \emph{Proofs of Space} \cite{Dziembowski2015} (also called \emph{Proof-of-Space and Time}). In this solution the computational puzzles are replaced with proofs that a given party contributed some disk space to the system. The only intensive computation happens in the setup phase, during which the user fills-in her disk with pseudorandom data. Once this is over, the user performs only occasional lightweight computations. The main problem in constructions comes from the so-called time-memory trade-offs that can be used for optimizing the space usage at a cost of slightly increasing the computing time. Luckily, as shown in \cite{Dziembowski2015} (and several subsequent papers, see, e.g., \cite{Ren2016,Pietrzak2019,Moran2019,Abusalah2017}) Proofs of Space that are secure against such attacks is possible. Several on-going blockchain projects are based on these ideas, including: \url{www.chia.net}, \url{filecoin.io}, and \url{spacemesh.io}. 

Recently, one of the aforementioned projects (\url{www.chia.net}) went live \cite{Hern}, and it is very instructive to observe its performance and the problems that appear during its operation. One of the notable problems that were discovered soon after Chia launch is that the setup procedure quickly destroys the SSD drives (since it requires multiple overwrites of a disk) \cite{Hern}. One of the research problems that we will work on is to construct a new Proof-of-Space algorithm that does not have this problem. We will continue monitoring the situation in this field and identify more interesting research problems to work on.




\subsection{Real-life side-channel security of blockchain wallets}\label{sec:side-channel}


Another important real-life problem with the vision of decentralizing internet services is that interacting with such protocols is more complicated than in the case of centralized solutions. Moreover, the decentralization makes it harder to revert the transactions that were posted by mistake or as a result of an attack. In several blockchains, reverting transactions is not possible at all (this is the case, e.g., in Bitcoin or Ethereum). Due to this problem, users are often discouraged to interact with blockchains using not secure devices such as PCs or smart phones. Instead, they often rely on the help of trusted servers (such as, e.g., \emph{Binance}, \emph{Coinbase}, or \emph{Kraken}), which is a solution that invalidates most of the benefits of the decentralization, and had disastrous effects in the past \cite{McMillan}. A much better solution is to use the so-called \emph{hardware wallets}, which are dedicated devices, protected against cyber-attacks. Several such commercial solutions are available on the market (e.g.~\emph{Ledger}, \emph{Trezor}, and \emph{KeepKey}). 

We will work on analyzing the security of the existing hardware wallets. In particular, we will be interested in their \emph{side-channel security}, i.e., security against attacks based on information such as power consumption or electromagnetic radiation. The PI has experience in the theoretical aspects of such attacks (this was the topic of his \emph{ERC Starting Grant}, 2010-2015). Also one of his on-going NCN grants is specifically focused on the foundations of wallet security. In IDEAS we will be more focused on the practical aspects of this problem. In particular, we plan to hire practitioners who are experts in the side-channel analysis to work on this topic. All the vulnerabilities found by us will be published under the \emph{responsible disclosure} model. We hope that the expertise gained in this way can potentially lead to constructions of new, more secure wallets. 

\section{Collaboration with other groups in IDEAS}

We plan to actively engage in collaboration with other groups of IDEAS. Some ideas for such collaborations include: using Machine Learning for verification of smart contracts (see Sec.~\ref{sec:verif}), for the side-channel analysis of hardware wallet security (see Sec.~\ref{sec:side-channel}), protocols for out-sourcing computationally intensive ML computations in a privacy-preserving and fair way. We will also engage in discussions related to the ethical aspects of AI, especially in topics that are also relevant to blockchain.



\section{Collaboration with scientific and industrial partners}

We also actively collaborate with universities and research institutions, such as TU Darmstadt, IST Austria, \'Ecole polytechnique, and others. This will be done by frequent mutual visits and online calls. We will also collaborate with the industrial partners. The PI also has a considerable experience in this area. In particular, he has been consulting for several cryptographic and FinTech companies in Poland and abroad, including  Billon sp.~z o.o, Brainbot Technologies, Enjin Pte Ltd., Monetha GmbH, Synerise SA, Polish Power Grid, and  Bitfold. We will also collaborate with practitioners from the blockchain space, in particular the Ethereum ecosystem (the PI has a research grant from the Ethereum Foundation, and gave talks on a number of events organized by this community). 

\section{Education and Dissemination}

We will be actively engaging in education. This will be primarily done by interaction with the MSc and PhD students of the Faculty of Mathematics, Informatics and Mechanics of the University of Warsaw, where the PI is a professor. We will also interact with students of other educational institutions in Warsaw, in particular the Warsaw University of Technology and Polish-Japanese Academy of Information Technology, and we will be open to collaboration with students from other Polish universities (in particular: the Jagiellonian University in Cracow). The PI has personal contacts in all of these institutions, which will facilitate access to good students. This will be done by teaching courses, giving seminars, and offering supervision of master and PhD theses. In particular, we will run a weekly group seminar that will be open to the general audience.

Our results will be communicated to the academic community via publications and talks at conferences, workshops, and schools. We will also disseminate knowledge to the society by participating in conferences with industry and policymakers in Poland and abroad (the PI has a vast experience with this, including talks at the Polish Ministry of Finance, the German Federal Bank, and conferences such as the \emph{European Forum for Science, Research and Innovation}, \emph{Digital Money \& Blockchain Forum}, \emph{Forum of Information and Communication Technology}, and others).

We will also maintain a webpage (possibly jointly with the webpage of the group at the University of Warsaw, \url{www.crypto.edu.pl}), and will be active in social media.




\printbibliography



\end{document}