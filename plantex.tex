\documentclass{article}

\usepackage{url}

\usepackage[backend=bibtex]{biblatex}
\addbibresource{plan}


\usepackage[bitstream-charter]{mathdesign}
\usepackage[T1]{fontenc}

\usepackage{fullpage}

\usepackage{titling}
\newcommand{\subtitle}[1]{%
	\posttitle{%
		\par\end{center}
	\begin{center}\large#1\end{center}
	\vskip0.5em}%
}

\title{\textbf{Blockchain and Smart Contract Group\\ IDEAS NCBR}}

\subtitle{\textbf{Research Plan}}

\date{\today}

\author{Stefan Dziembowski}

\begin{document}
	
\maketitle


\emph{Distributed cryptography} is a technology developed since the 1980s \cite{DBLP:conf/focs/Yao86,DBLP:conf/stoc/GoldreichMW87,DBLP:conf/crypto/ChaumCD87,DBLP:conf/stoc/Ben-OrGW88}, initially mostly by the theory community building upon the ideas from the area of distributed algorithms such a the \emph{Byzantine consensus} \cite{DBLP:journals/toplas/LamportSP82}. Its main advantage is that it reduces the trust assumptions in network environments: in a distributed cryptography protocol typically only a certain fraction of participants needs to be honest. This is in contrast with the \emph{centralized} (i.e.: ``non-distributed'') solutions where the users usually need to entirely trust a single participant (often called a \emph{trusted server}). Unfortunately, for a long time distributed cryptography has not been used  massively in practice, mostly due to the fact that the using it introduced a large computational overhead compared to the centralized solutions. In other words: typical users where not ready to pay (in terms of computing time, equipment costs, etc.) for the security benefits of this technology.

This situation has changed recently. One reason for this is a general erosion of trust in single-server solutions, partly due to Edward Snowden's revelations \cite{enwiki:1029775076}, and partly because of cases like the \emph{Facebook-Cambridge Analytica data scandal} \cite{enwiki:1029165846} which led to growing understanding that the centralized solutions give too much power to the operators of the servers.

 \emph{Blockchain technology}, introduced around one decade ago by a pseudonymous author in \cite{nakamoto2008bitcoin}, revolutionized the way the society thinks about the 

The group will to a large extend build upon the results of the basic research 
by moving them closer to real-life applications.

Proof of concept

Collaboration




\section{Practice-oriented formal modeling and verification of smart contracts protocols}

\section{Proof of Space in practice}

\section{Real life side-channel security of blockchain wallets}


\section{Education}



\printbibliography



\end{document}